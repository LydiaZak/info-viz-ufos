\documentclass[journal]{vgtc}                % final (journal style)
%\documentclass[review,journal]{vgtc}         % review (journal style)
%\documentclass[widereview]{vgtc}             % wide-spaced review
%\documentclass[preprint,journal]{vgtc}       % preprint (journal style)

%% Uncomment one of the lines above depending on where your paper is
%% in the conference process. ``review'' and ``widereview'' are for review
%% submission, ``preprint'' is for pre-publication, and the final version
%% doesn't use a specific qualifier.

%% Please use one of the ``review'' options in combination with the
%% assigned online id (see below) ONLY if your paper uses a double blind
%% review process. Some conferences, like IEEE Vis and InfoVis, have NOT
%% in the past.

%% Please note that the use of figures other than the optional teaser is not permitted on the first page
%% of the journal version.  Figures should begin on the second page and be
%% in CMYK or Grey scale format, otherwise, colour shifting may occur
%% during the printing process.  Papers submitted with figures other than the optional teaser on the
%% first page will be refused. Also, the teaser figure should only have the
%% width of the abstract as the template enforces it.

%% These few lines make a distinction between latex and pdflatex calls and they
%% bring in essential packages for graphics and font handling.
%% Note that due to the \DeclareGraphicsExtensions{} call it is no longer necessary
%% to provide the the path and extension of a graphics file:
%% \includegraphics{diamondrule} is completely sufficient.
%%
\ifpdf%                                % if we use pdflatex
  \pdfoutput=1\relax                   % create PDFs from pdfLaTeX
  \pdfcompresslevel=9                  % PDF Compression
  \pdfoptionpdfminorversion=7          % create PDF 1.7
  \ExecuteOptions{pdftex}
  \usepackage{graphicx}                % allow us to embed graphics files
  \DeclareGraphicsExtensions{.pdf,.png,.jpg,.jpeg} % for pdflatex we expect .pdf, .png, or .jpg files
\else%                                 % else we use pure latex
  \ExecuteOptions{dvips}
  \usepackage{graphicx}                % allow us to embed graphics files
  \DeclareGraphicsExtensions{.eps}     % for pure latex we expect eps files
\fi%

%% it is recomended to use ``\autoref{sec:bla}'' instead of ``Fig.~\ref{sec:bla}''
\graphicspath{{figures/}{pictures/}{images/}{./}} % where to search for the images

\usepackage{microtype}                 % use micro-typography (slightly more compact, better to read)
\PassOptionsToPackage{warn}{textcomp}  % to address font issues with \textrightarrow
\usepackage{textcomp}                  % use better special symbols
\usepackage{mathptmx}                  % use matching math font
\usepackage{times}                     % we use Times as the main font
\renewcommand*\ttdefault{txtt}         % a nicer typewriter font
\usepackage{cite}                      % needed to automatically sort the references
\usepackage{tabu}                      % only used for the table example
\usepackage{booktabs}                  % only used for the table example
%% We encourage the use of mathptmx for consistent usage of times font
%% throughout the proceedings. However, if you encounter conflicts
%% with other math-related packages, you may want to disable it.

%% In preprint mode you may define your own headline.
%\preprinttext{To appear in IEEE Transactions on Visualization and Computer Graphics.}

%% If you are submitting a paper to a conference for review with a double
%% blind reviewing process, please replace the value ``0'' below with your
%% OnlineID. Otherwise, you may safely leave it at ``0''.
\onlineid{0}

%% declare the category of your paper, only shown in review mode
\vgtccategory{Research}
%% please declare the paper type of your paper to help reviewers, only shown in review mode
%% choices:
%% * algorithm/technique
%% * application/design study
%% * evaluation
%% * system
%% * theory/model
\vgtcpapertype{please specify}

%% Paper title.
\title{Visualization of Reported UFO Sightings}

%% This is how authors are specified in the journal style

%% indicate IEEE Member or Student Member in form indicated below
\author{Peter Bernstein, Emily Dutile, Abigail Skelton, and Lydia Zakynthinou}
\authorfooter{
%% insert punctuation at end of each item
\item
Emily Dutile and Abigail Skelton are graduate students at Northeastern University. E-mail: \{dutile.e, skelton.a\}@husky.neu.edu.
\item
Peter Bernstein and Lydia Zakynthinou are PhD students at Northeastern University. E-mail: \{bernstein.p, zakynthinou.l\}@husky.neu.edu.
}


%other entries to be set up for journal
\shortauthortitle{Biv \MakeLowercase{\textit{et al.}}: Visualization of Reported UFO Sightings}
%\shortauthortitle{Firstauthor \MakeLowercase{\textit{et al.}}: Paper Title}

%% Abstract section.
\abstract{Reports of unidentified flying objects (\textit{UFOs}) have sparked amateur research, government investigations,
 and large popular interest in the last five decades. Most reported UFOs are later identified as natural phenomena or
 conventional objects. However, there is a considerable body of reports about objects that are not identified, which are 
 usually perceived as claims of observations of
 extraterrestrial crafts, thus raising questions about life on other planets and the likelihood of
 extraterrestrials visiting Earth. Although scientists in their majority have naturally greeted the topic with
 skepticism, it is widely recognized that answering these questions would be, among other things, of great
 scientific importance and a big step towards understanding the universe. We provide an interactive
 visualization of the reported UFO sightings in the United States in the period of 1964-2017,
 aiming to help any interested user explore these sightings.
} % end of abstract

%% Keywords that describe your work. Will show as 'Index Terms' in journal
%% please capitalize first letter and insert punctuation after last keyword
\keywords{Visualization, Map, Interactivity, Unidentified Flying Object, UFO, United States.}

%% ACM Computing Classification System (CCS). 
%% See <http://www.acm.org/class/1998/> for details.
%% The ``\CCScat'' command takes four arguments.

\CCScatlist{ % not used in journal version
 \CCScat{K.6.1}{Management of Computing and Information Systems}%
{Project and People Management}{Life Cycle};
 \CCScat{K.7.m}{The Computing Profession}{Miscellaneous}{Ethics}
}

%% Uncomment below to include a teaser figure.
\teaser{
  \centering
  \includegraphics[width=\linewidth]{ufoimage}
  \caption{UFOs are ``proved beyond reasonable doubt"��: A rotating ``��glowing aura"� traveling at high speeds that was captured from a Navy F/A-18 Super Hornet. \cite{pentagon}}
	\label{fig:teaser}
}

%% Uncomment below to disable the manuscript note
\renewcommand{\manuscriptnotetxt}{}

%% Copyright space is enabled by default as required by guidelines.
%% It is disabled by the 'review' option or via the following command:
%\nocopyrightspace

\vgtcinsertpkg

%%%%%%%%%%%%%%%%%%%%%%%%%%%%%%%%%%%%%%%%%%%%%%%%%%%%%%%%%%%%%%%%
%%%%%%%%%%%%%%%%%%%%%% START OF THE PAPER %%%%%%%%%%%%%%%%%%%%%%
%%%%%%%%%%%%%%%%%%%%%%%%%%%%%%%%%%%%%%%%%%%%%%%%%%%%%%%%%%%%%%%%%

\begin{document}

%% The ``\maketitle'' command must be the first command after the
%% ``\begin{document}'' command. It prepares and prints the title block.

%% the only exception to this rule is the \firstsection command
\firstsection{Introduction}

\maketitle

%% \section{Introduction} %for journal use above \firstsection{..} instead
An unidentified flying object (UFO) is a perceived object in the sky that is not readily identified. The term UFO (initially, \textit{UFOB}) appeared in 1953 when the United States Air Force used it to describe ``any airborne object which by performance, aerodynamic characteristics, or unusual features, does not conform to any presently known aircraft or missile type, or which cannot be positively identified as a familiar object" \cite{ufowiki}. Since the 1950s, UFOs have become a major subject of interest in the popular culture and an inspiration for several movies and books \cite{history}. The reason for this is the fact that UFOs are linked to suspected extraterrestrial aircrafts, and if this were true it would suggest that life exists beyond Earth and even more that extraterrestrials visit our planet.\\
Although UFOs are largely connected to this theme, it is true that for most of the reported cases the objects are identified to be ordinary or to be caused by a natural phenomenon after careful investigation. Most commonly, UFOs are identified to actually be astronomical objects, aircrafts, balloons (e.g. weather, research balloons), atmospheric or light phenomena (e.g. clouds, mirages), other atmospheric objects (e.g. birds), or, in some rare cases, hoaxes. The percentage of reports of objects that remain unidentified lies between 5\% and 20\%  \cite{ufowiki}. However, this is a large enough percentage to spur a significant amount of government research and funds as well as scientific attention.

Of the most recently revealed government programs on UFOs is the U.S. Defense Department's \textit{Advanced Aerospace Threat Identification Program} \cite{pentagon}. The program, which was led by military intelligence official Luis Elizondo, investigated evidence of UFOs and extraterrestrial life from 2007 to 2012, with an annual budget of 22 million dollars. In 2012 it was shuttered due to a change in the department's funding priorities, but Elizondo oversaw UFO investigations until this past October. He is convinced that this is a matter of importance and even contends that UFOs are ``proved beyond reasonable doubt" (Fig.1). As for the scientific community, the recent \textit{Breakthrough Listen Program} \cite{listen}, located in the Astronomy Department of the University of California, Berkeley, is the most scientifically comprehensive search for intelligent extraterrestrial communications in the Universe to date \cite{listenwiki}. It counts 100 million dollars in funding and it became more publicly known due to the statements of renowned physicist Stephen Hawking on alien life and his support of the program \cite{hawking}.

It is clear that the subject is controversial but, at the same time, most would agree that it requires research and attention. Therefore, it is important that there is a general awareness of the subject to the public. To this end, we created an interactive web-based visualization that allows the user to explore reports of sightings in the United States. It is intended to be pleasing and to give an overview of the reported sightings in the United States throughout the years, as well as the ability to drill down in order to explore characteristics of more specific areas and sightings.

The data we are using is posted on the website of the National UFO Report Center, whose head is Peter Davenport. The Center's website provides an online form as well as a hotline for reports of UFO sightings and these reports are annotated by Peter Davenport himself before they are posted in the database. Each report includes the date and time of the sighting, its duration, shape, location, and description summary (possibly including notes of Peter Davenport in double parentheses), as well as the date the report was posted.

\section{Related Works}
The dataset we are using has been popular on Kaggle which led two data visualization experts, Pooja Gandhi and Adam Crahen, to use it in their DuoDare project on their DataDuo blog \cite{dataduo}. The DuoDare was a project where each month one of the two experts would choose a dataset and call the other on a battle for the best visualization. The two visualizations that the experts came up with for this dataset included, among others, interactive maps, area charts, bar charts, and calendar heatmaps. Compared to these visualization, ours gives the user the chance to look into a state in more detail and learn information about a selected group of sightings.

\section{Process}

\subsection{Task Analysis}

\section{Design}

(i.e., visual encoding and interaction justifications)

\section{Discussion}



\section{Conclusion}




\begin{thebibliography}{9}

\bibitem{blocks} 
Mike Bostock.
\textit{Blocks}. 
Click-to-Zoom via Transform,
\\\texttt{https://bl.ocks.org/mbostock/2206590}

\bibitem{ufoimage} 
Newsweek.
\textit{UFO existence proven beyond reasonable doubt says former head of Pentagon alien program}.
\\\texttt{http://www.newsweek.com/ufo-existence-proven-beyond\\-reasonable-doubt-says-former-head-pentagon-alien-758293}

\bibitem{dataduo}
DataDuo - DuoDare Project on UFO sightings
\\\texttt{https://thedataduo.com/2017/11/11/duodare-3\\-ufo-sightings/}

\bibitem{ufowiki}
Wikipedia.
\textit{Unidentified flying object}.
\\\texttt{https://wikipedia.org/wiki/Unidentified\_flying\_object}

\bibitem{history}
History.
\textit{History of UFOs}.
\\\texttt{https://www.history.com/topics/history-of-ufos}

\bibitem{pentagon}
The Guardian.
\textit{Pentagon admits running secret UFO investigation for five years}.
\\\texttt{https://www.theguardian.com/world/2017/dec/17/\\pentagon-admits-running-secret-ufo-investigation-for-five-years}

\bibitem{listen}
The Guardian.
\textit{Why we keep scanning the skies for signs of alien intelligence}.
\\\texttt{https://www.theguardian.com/science/2017/dec/22/\\why-we-keep-scanning-the-skies-for-signs-of-alien-intelligence}

\bibitem{listenwiki}
Wikipedia.
\textit{Breakthrough Listen}.
\\\texttt{https://en.wikipedia.org/wiki/Breakthrough\_Listen}

\bibitem{hawking}
Newsweek.
\textit{Stephen Hawking on alien life, extraterrestrials and the possibility of UFOs visiting Earth}.
\texttt{http://www.newsweek.com/stephen-hawking-death-alien\\-contact-aliens-theory-ufo-search-breakthrough-844920}

\end{thebibliography}





%% if specified like this the section will be committed in review mode


%\bibliographystyle{abbrv}
\bibliographystyle{abbrv-doi}
%\bibliographystyle{abbrv-doi-narrow}
%\bibliographystyle{abbrv-doi-hyperref}
%\bibliographystyle{abbrv-doi-hyperref-narrow}

\bibliography{template}
\end{document}

