\documentclass{neu_handout}
\usepackage{url}
\usepackage{amssymb}
\usepackage{amsmath}
\usepackage{marvosym}
\usepackage{graphicx}
\usepackage[pdftex]{graphicx}
\usepackage{subfigure}
\graphicspath{ {images/} }
\everymath{\displaystyle}

% Professor/Course information
\title{Milestone 1 - UFOs}
\author{Abby, Emily, Lydia and Peter}
\date{March 2018}
\course{CS 7295}{Information Visualization}

\begin{document}

\section*{1 Data acquisition and clean-up}

\textbf{How dirty was the data and what kind of clean-up was required? How
easy was it to download / use API? What is the format of the data? What are the
items and attributes in your dataset? What types of data are you working with
(categorical, ordinal, quantitative)?}\\


\section*{2 Exploring the data}

\textbf{Any interesting
trends/observations? Any evidence of missing or “dirty” data? Any features or
observations in the data that are confusing or unexpected?}\\



\section*{3 Interview}

\textbf{Write a paragraph: how did the interview go? What did you learn? What were you surprised by during the interview? Has the interview changed your motivating questions?}\\ 

After some initial trouble communicating with the director of the National UFO database, and trouble with sourcing a line of communication, we had a solid half hour interview with Peter Davenport, Director of the National UFO Reporting Center. Davenport discussed with us many things regarding his database, most notable of which was his role in establishing the data collection method that the Center uses now. Davenport modernized the collection and organization of the data, both retroactively and for use in the future. I was surprised by how much the tone and attitude of the interview subject changed over the course of the scheduling process and interview; initially, Davenport was skeptical of our use of his data and interview. During the interview, it came to light that this was due to him being burned before by major publications and national news outlets. Our interview was incredibly informative but I do not believe that it changed our line of questioning or reasoning in this project. 




\section*{4 Task analysis}

\textbf{Create a full list of “domain” tasks (i.e., what are the tasks a
user wants to accomplish with the data using your visualization) and translate
these into high/mid/low level tasks.}

\subsection*{4.1 Domain Tasks}

\subsection*{4.2 Sketches}


\end{document}
